\section{Przemysław Maresz}
\label{przemyslaw_maresz}

Photo of Sundayan proboscis monkey (see Figure~\ref{fig:nosacz}).
\begin{figure}[htbp]
    \centering
    \includegraphics[width=0.7\textwidth]{pictures/Nosacz.jpg}
    \caption{Pjoter ty byś Cormena poczytał a nie w te komputery grasz}
    \label{fig:nosacz}
\end{figure}

W poszczególnych rozdziałach omawiamy algorytmy, metody ich projektowania, dziedziny ich zastosowań lub 
inne pokrewne zagadnienia. 

Algorytmy sa zapisane w pseudojezyku programowania, który jest zrozumiały dla 
każdego, kto ma choć odrobine praktyki programistycznej. Ksiażka zawiera ponad 260 rysunków obrazujacych 
działanie algorytmów. Ponieważ naszym kryterium optymalności algorytmów jest ich złożoność , przedstawiamy 
szczegółowa analize czasu ich działania.

\textbf{Thomas H. Cormen, Charles E. Leiserson
, Ronald L. Rivest}

\begin{figure}[htbp]
Najdłuższe rzeki świata:
\begin{enumerate}
    \item Nil
    \item Amazonka
    \item Jangcy
\end{enumerate}
Środki transportu:
\begin{itemize}
    \item samochód
    \item motocykl
    \item rower
\end{itemize}
\end{figure}

\begin{table}[htbp]
    \centering
    \begin{tabular}{|c|c|c|c|}
    \hline
                 &Euro   &Dolary &Funty \\ \hline
wartość w PLN   &4,62 zł    &3,98 zł    &5,47 zł \\ \hline

    \end{tabular}
    \caption{kurs walut}
    \label{tab:przem_table1}
\end{table}

Here is a math equation: 
\[ (a + b)^2 = a^2 + 2ab + b^2 \]

Another one is here: $ (a + b)(a - b) = a^2 - b^2 $